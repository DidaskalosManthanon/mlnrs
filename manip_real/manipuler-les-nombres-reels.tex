% Created 2023-08-19 Sat 20:25
% Intended LaTeX compiler: pdflatex
\documentclass[11pt]{article}
\usepackage[utf8]{inputenc}
\usepackage[T1]{fontenc}
\usepackage{graphicx}
\usepackage{longtable}
\usepackage{wrapfig}
\usepackage{rotating}
\usepackage[normalem]{ulem}
\usepackage{amsmath}
\usepackage{amssymb}
\usepackage{capt-of}
\usepackage{hyperref}
\usepackage{color}
\usepackage{minted}
\usepackage{parskip}
\usepackage{tikz}
\author{Laurent Garnier}
\date{\today}
\title{Manipuler les nombres réels}
\hypersetup{
 pdfauthor={Laurent Garnier},
 pdftitle={Manipuler les nombres réels},
 pdfkeywords={},
 pdfsubject={},
 pdfcreator={Emacs 28.1 (Org mode 9.5.2)}, 
 pdflang={English}}
\begin{document}

\maketitle
\tableofcontents


\section{À qui s'adresse ce livre}
\label{sec:org835d39c}

Ce livre s'adresse initialement aux élèves de seconde du système
scolaire français. Mais il sera probablement très utile à un élève
de fin de collège qui voudrait assurer son entrée au lycée. Il sera
également très utile à un élève de première qui voudrait s'assurer
qu'il maîtrise bien les notions de la classe de seconde. Enfin un
quatrième public qui pourrait directement être visé est celui des
enseignants, profs particuliers ou encore tuteurs (professionnels ou
amateurs) qui souhaiteraient consulter des ressources utiles et
utilisables pour aider les lycéens.

De façon plus générale, même si je reprends exactement la structure
du programme officiel que vous pouvez consulter ici :
\url{https://eduscol.education.fr/document/24553/download}

Il me semble que ce livre pourrait être utile à quiconque souhaite
approcher les mathématiques de façon honnête, rigoureuse et avec un
soucis de compréhension. Tout en respectant le niveau du programme
j'essaierai dès que possibles de prolonger la perspective.

Vous pouvez d'ores et déjà mettre en pratique de façon interactive
toutes les notions abordées grâce aux codes Python accessible sur
Google Colab à l'adresse communiquée dans la section concernée.

Bonne lecture.

\section{Contenus}
\label{sec:org98ac306}
\subsection{Ensemble \(\mathbb{R}\) des nombres réels, droite numérique}
\label{sec:org0c17de3}

Savez-vous résoudre l'équation \(x - 1 = 0\) ? Je vous laisse
réfléchir et je vous donnerai la solution plus tard.

Imaginez que vous êtes en train de jouer à votre jeu vidéo
préféré. Il vous reste 3 points de vie, combien pouvez-vous en
perdre avant de mourir (dans le jeu) ? Idem, je vous laisse
réfléchir.

En général, lorsque vous entrez en seconde vous êtes âgé de 15 ans,
du coup combien d'années vous séparent de votre majorité ?

Ces trois questions traitent du même genre de problèmes, la
résolution des équations algébriques du premier degré impliquant
les nombres les plus simples que vous connaissez à savoir les
nombres entiers naturels. Pour la faire simple il s'agit des
nombres utiles pour compter le nombre de doigts que vous possédez,
le nombre d'enfants dans une famille et toutes les choses qu'on
peut dénombrer de zéro à l'infini.

On note l'ensemble des entiers naturels \(\mathbb{N} = \{0, 1, 2, 3,
   \dots \}\).

\begin{align}
x - 1 &= 0 \Rightarrow x = 1\in\mathbb{N}\\
3 - x &= 0 \Rightarrow x = 3\in\mathbb{N}\\
15 + x&= 18\Rightarrow x = 3\in\mathbb{N}
\end{align}

Mais ces nombres sont insuffisants pour exprimer des températures
par exemple dans une ville d'Europe du Nord comme Saint Pétersbourg
en Russie ou Oslo en Norvège. En fait, même en France
métropolitaine si vous allez en Savoie vous observerez des
températures négatives en hiver et positive en été. Le signe de ces
températures est \emph{relatif} à la position du zéro.

Les nombres entiers relatifs sont constitués des nombres entiers
naturels (les nombres positifs qu'on a vu précédemment) auquels on
ajoute leurs symétriques par rapport à zéro.

On les notes \(\mathbb{Z} = \{\dots, -2, -1, 0, 1, 2, \dots\}\).

Ce sont également ces nombres qu'on utilise dans un ascenseur pour
indiquer si on va au sous-sol (nombre négatif) ou aux étages
supérieurs (nombres positifs).

Dans le langage de programmation Python les nombres entiers
relatifs sont représentés par le type \texttt{int} pour \emph{integer} (qui
veut dire entier relatif en anglais).

Mais ces nombres sont encore en nombre insuffisant pour exprimer
par exemple les unités monétaires. Quand vous achetez une baguette,
du carburant, des fournitures scolaires ou quoique ce soit ; la
plupart du temps vous utilisez des nombres à virgules qu'on appelle
les nombres décimaux.

Contrairement aux deux ensembles précédents, l'ensemble des nombres
décimaux ne peut pas être représenté de façon explicite avec des
points de suspension, il doit être définit en
\emph{compréhension}. Définir un ensemble en compréhension signifie que
vous devez expliquer la règle de construction.

On note \(\mathbb{D} = \left\{ \dfrac{a}{10^n}\,|\,(a,
   n)\in\mathbb{Z}^2\right\}\)

Commençons par la fin \((a, n)\in\mathbb{Z}^2\) signifie que \(a\) et
\(n\) sont tous les deux des entiers relatifs. Ainsi, les décimaux
sont les nombres fractionnaires dont le numérateur est un entier
relatif et le dénominateur un puissance de 10. Donnons quelques
exemples :
\begin{align}
(a, n) &= (1, 2)\Rightarrow \dfrac{1}{10^2} = 0,01\\
(a, n) &= (3, 4)\Rightarrow \dfrac{3}{10^4} = 0,0003\\
(a, n) &= (2, -1)\Rightarrow \dfrac{2}{10^{-1}} = 20\\
(a, n) &= (3, 0)\Rightarrow \dfrac{3}{10^0} = 3\\
(a, n) &= (-1, 2)\Rightarrow \dfrac{-1}{10^2} = -0,01
\end{align}

Dans la programmation Python on parle du type \texttt{float} pour
\emph{floating point number} ou nombre à virgule flottante en bon
français (puisque les anglo-saxons utilisent un point comme
séparateur décimal). En prenant \(n = 0\) on obtient tous les entiers
relatifs. Mais ce n'est pas suffisant pour décrire le réel (en
théorie, parce qu'en pratique les ordinateurs s'en contentent).

Imaginez que vous avez un budget pour une semaine et que vous
souhaitiez allouer une part équitable pour chaque jour. Et bien
sachez que le nombre \(\dfrac{1}{7}\) n'est pas un décimal, pas plus
que le nombre \(\dfrac{1}{6}\). En effet si vous posez la division ou
que vous utilisez une machine (calculatrice, ordinateur, tablette
ou téléphone peu importe) vous verrez que \(\dfrac{1}{7} \simeq
   0,142857\dots\) et que cette séquence \(142857\dots\) se répètent à
l'infini. De même, de façon plus simple, si vous prenez
\(\dfrac{1}{6} \simeq 0,16\dots\) les 6 se répètent à l'infini.

Posons \(x = 0,16\dots 6\dots\) où les 6 se répètent à
l'infini. Multiplions par 100 ainsi on obtient \(100x = 16,6\dots
   6\dots\). Multiplions par 10 ainsi on obtient \(10x = 1,6\dots
   6\dots\). Faisons maintenant la soustraction \(90x = 15\) d'où \(x =
   \dfrac{15}{90} = \dfrac{3\times 5}{2\times 3\times 3\times 5} =
   \dfrac{1}{6}\). Le nombre \(\dfrac{1}{6}\) est bien un nombre qui sort
de la définition des décimaux et justifie une nouvelle classe de
nombres, les nombres rationnels (ratio en latin veut dire
quotient).

Les nombres rationnels sont notés \[\mathbb{Q} =
   \left\{\dfrac{a}{b}\,|\,(a,
   b)\in\mathbb{Z}\times\mathbb{N}^*\right\}\]

Ici le numérateur \(a\) est un entier relatif et le dénominateur \(b\)
est un entier naturel strictement positif parce qu'on ne peut pas
diviser par zéro.

Prouvons ce dernier résultat. Imaginez que l'on puisse trouver un
nombre entier \(a\) qui soit divisible par zéro. Alors il existerait
un entier \(q\) tel que \(a = 0\times q\). Le problème c'est que zéro
est un élément absorbant, tel le trou noir de la multiplication,
tout nombre multiplié par zéro donne zéro. La division par zéro n'a
donc pas de sens.

Donnons quelques exemples de nombre rationnels :

\begin{align}
(a, b) &= (1, 2)\Rightarrow \dfrac{1}{2} = 0,5\in\mathbb{D}\\
(a, b) &= (8, 9)\Rightarrow \dfrac{8}{9} = 0,8\dots 8\dots\in\mathbb{Q}\\
(a, b) &= (3, 4)\Rightarrow \dfrac{3}{4} = 0,75\in\mathbb{D}\\
(a, b) &= (19, 11)\Rightarrow \dfrac{19}{11} = 1,72\dots \in\mathbb{Q}\\
(a, b) &= (2, 7)\Rightarrow \dfrac{2}{7} = 0,285714\dots \in\mathbb{Q}
\end{align}


Les nombres rationnels sont très intéressants mais ils restent
insuffisants pour capturer le réel. Imaginez que l'écran de votre
ordinateur portable soit un rectangle de longueur 16 cm pour une
largeur de 9 cm (c'est le fameux format 16/9). Alors pour calculer
la diagonale nous allons utiliser le théorème de Pythagore :
\begin{align*}
d^2 &= 16^2 + 9^2\\
d^2 &= 256 + 81\\
d^2 &= 337\\
d &= \sqrt{337} \simeq 18,36
\end{align*}

Tout d'abord le nombre 337 est un nombre premier (prouvez-le, c'est
un bon exercice). Et ensuite la racine carrée d'un nombre premier
ne peut pas s'écrire sous la forme d'un nombre rationnel.

Pour des raisons de concordance avec le programme de seconde je
vous propose de démontrer que \(\sqrt{2}\) n'est pas un nombre
rationnel et pour le résultat (plus général) énoncé précédemment,
je vous renvoie aux approfondissements (mais vous pouvez déjà
commencer à chercher).

Pour démontrer que \(\sqrt{2}\) n'est pas un nombre rationnel on va
utiliser plusieurs types de raisonnement. Tout d'abord nous allons
utiliser le raisonnement par l'absurde. Le raisonnement par
l'absurde repose sur un principe de la logique classique qui dit
qu'une affirmation est soit vraie soit fausse. En logique
mathématique on ne s'occupe que de ce genre d'affirmation donc nous
sommes dans un système \textbf{beaucoup} plus simple que la réalité. Du
coup, si le contraire de ce que je veux montrer est faux alors ce
que je veux montrer est vrai. Prenons un exemple concret un peu
simpliste, si ce n'est pas le matin alors c'est l'après-midi. Si
vous n'êtes pas majeur alors vous êtes mineur. Ce genre de logique
fonctionne bien avec les systèmes binaires.

Revenons à nos moutons, on veut montrer que \(\sqrt{2}\) n'est pas
rationnel donc on va supposons le contraire et montrer que ce
contraire est faux donc que notre affirmation sera vraie.

Supposons donc par l'absurde que \(\sqrt{2}\) soit rationnel donc
d'après la définition de \(\mathbb{Q}\) il devrait exister un entier
relatif \(a\) et un entier naturel strictement positif \(b\) tels que
\(\sqrt{2} = \frac{a}{b}\). Comme toute fraction peut se ramener
(après simplification) à une fraction irréductible on peut donc
considérer que \(a\) et \(b\) n'ont pas de diviseurs communs et que la
fraction est irréductible. Dire que \(a\) et \(b\) n'ont pas de
diviseurs communs revient à dire que leur pgcd vaut un c'est-à-dire
\(pgcd(a, b) = 1\). Le PGCD est le Plus Grand Commun Diviseur. Bon on
devrait l'appeler le PGDC parce qu'en français on dit diviseur
commun car l'adjectif vient après le nom mais en anglais ils disent
\emph{greatest common divisor} et bien souvent les gens importent les
concepts anglophones en oubliant les règles du français c'est pour
ça que sur certaines calculatrices et logiciels vous pouvez voir
GCD ou gcd.

Faisons un bref récapitulatif :
\begin{itemize}
\item Objectif : Montrer que \(\sqrt{2}\) n'est pas rationnel
\item Méthode : Raisonnement par l'absurde, on pose \(\sqrt{2} =
     \frac{a}{b}\) et avec \(pgcd(a, b) = 1\)
\end{itemize}

Du coup, on va élever au carré ce qui nous donne \(2 =
   \frac{a^2}{b^2}\) et qu'on peut mettre sous la forme \(a^2 = 2b^2\).

On peut classer les nombres entiers en deux catégories :
\begin{enumerate}
\item Les nombres pairs de la forme \(p = 2n\) où \(n\) est un entier.
\item Les nombres impairs de la forme \(i = 2n + 1\) où \(n\) est un
entier.
\end{enumerate}

On en déduit que le nombre \(a^2 = 2b^2\) appartient à la première
catégorie avec \(n = b^2\).

Désormais on va avoir besoin d'un résultat intermédiaire qui va
faire intervenir un autre type de raisonnement.

Voici le résultat intermédiaire qu'on veut montrer : \emph{si le carré
d'un nombre est pair alors le nombre initial est aussi pair}.

On va formaliser cette affirmation :
\begin{itemize}
\item A = "le carré d'un nombre est pair"
\item B = "le nombre initial est pair"
\item \(P = A\Rightarrow B\)
\end{itemize}

Le nouveau type de raisonnement qu'on va utiliser s'appelle le
raisonnement par contraposée.

Prenons un exemple concret fictif (et simpliste). Imaginez que
chaque fois qu'il pleut je prenne mon parapluie (sans \textbf{aucune}
exception). Si vous me croisez \textbf{sans} parapluie alors vous en
déduirez qu'il ne pleut pas. C'est une proposition logiquement
équivalente à la première formulation.

\begin{itemize}
\item A = "Il pleut"
\item B = "Je prend mon parapluie"
\item \(P = A\Rightarrow B\)
\item non B = "Je n'ai pas de parapluie"
\item non A = "Il ne pleut pas"
\item \(CP(P) = non(B)\Rightarrow non(A)\)
\end{itemize}

Revenons à nos nombres entiers. La contraposée de : "si le carré
d'un nombre est pair alors le nombre initial est aussi pair" est
"Si le nombre initial est impair alors son carré aussi."

Décomposons :
\begin{itemize}
\item A = "le carré d'un nombre est pair"
\item B = "le nombre initial est pair"
\item \(P = A\Rightarrow B\)
\item non B = "le nombre initial est impair"
\item non A = "le carré est impair"
\item \(CP(P) = non(B)\Rightarrow non(A)\)
\end{itemize}

Ce raisonnement nous permet de ramener la proposition à montrer à
une proposition plus simple et qu'on peut montrer directement.

Si le nombre initial est impair alors il est de la forme \(i = 2n +
   1\). Il vient que son carré satisfait :
\begin{align*}
i^2 &= (2n + 1)^2\\
i^2 &= (2n)^2 + 2\times 2n\times 1 + 1^2\\
i^2 &= 2(2n^2 + 2n) + 1
\end{align*}

On a bien obtenu un nombre impair. Par équivalence logique la
proposition initiale est vraie.

Faisons un bilan d'étape.
\begin{itemize}
\item Objectif final : Montrer que \(\sqrt{2}\) n'est pas rationnel
\item Méthode : Raisonnement par l'absurde, on pose \(\sqrt{2} =
     \frac{a}{b}\) et avec \(pgcd(a, b) = 1\)
\item Objectif intermédiaire : \(a^2 = 2b^2\) est pair donc \(a\) est pair.
\item Méthode pour l'objectif intermédiaire : raisonnement par
contraposée, si \(a\) est impair alors \(a^2\) aussi
\end{itemize}

À ce stade nous avons bien montré que \(a^2 = 2b^2\) entraîne que
\(a = 2n\). De cette nouvelle écriture on tire \(a^2 = 4n^2\) et on
peut établir une nouvelle égalité \(2b^2 = 4n^2\) qui donne \(b^2 =
   2n^2\). Par conséquent, en utilisant notre résultat intermédiaire,
on constate que \(b\) est aussi pair.

Mais si \(a\) et \(b\) sont pairs alors ils sont divisibles par deux ce
qui contredit \(pgcd(a, b) = 1\). Ce résultat est absurde donc
l'hypothèse initiale aussi et finalement \(\sqrt{2}\) n'est pas un
rationnel. CQFD

Bravo à vous si vous avez tout compris. Pour les autres,
rassurez-vous, c'est normal que ça chauffe dans votre tête. Vous
pouvez déjà revoir cette démonstration en vidéo ici :
\url{https://youtu.be/R\_L4NEgIxPM}

Le raisonnement par l'absurde consiste à partir d'une hypothèse
contraire à celle qu'on souhaite montrer et ensuite à enchaîner des
déductions logiques parfaitement valides jusqu'à aboutir à une
contradiction. La chaîne de déductions peut parfois s'avérer (très)
longue et impliquer des résultats intermédiaires nécessitant à leur
tour (éventuellement) d'autres types de raisonnement.

Dans certains ouvrages, certaines personnes préfèrent admettre les
résultats intermédiaires. Dans l'absolu il serait impossible de
tout démontrer parce qu'on ne pourrait jamais avancer. Néanmoins il
est utile de garder à l'esprit qu'en mathématiques le but du jeu
est de démontrer les affirmations sans jamais les prendre pour
argent comptant.

Finalement, puisque \(\sqrt{2}\) n'est pas un nombre rationnel il
doit bien appartenir à un ensemble de nombre et cet ensemble est
l'ensemble des nombres réels et on le note \(\mathbb{R}\).

Nous avons ainsi la chaîne d'inclusions ensemblistes :

\[\mathbb{N}\subset\mathbb{Z}\subset\mathbb{D}\subset\mathbb{Q}\subset\mathbb{R}\]

Cet ensemble de nombres correspond à l'infinité des points sur une
droite et c'est pour cette raison qu'on représentera souvent les
nombres réels par une droite graduée qu'on appellera la droite des
réels avec 0 au milieu et qui se prolonge vers \(+\infty\) à droite
et \(-\infty\) à gauche.



\subsection{Intervalles de \(\mathbb{R}\). Notions \(+\infty\) et \(-\infty\).}
\label{sec:org108e677}

Jusqu'à présent nous avons traité essentiellement des égalités (une
équation est une égalité). Mais dans de nombreux cas il est
intéressant et utile de traiter des inégalités. Donnons quelques
exemples issus du monde réel de la vie de tous les jours :
\begin{enumerate}
\item Il faut être âgé de plus de 14 ans pour conduire un véhicule de
type mobilette ou scooter.
\item Il faut être âgé entre 12 et 25 ans pour bénéficier du tarif
jeune à la SNCF.
\item Les films pornographiques sont interdits aux mineurs donc aux
gens qui ont un âge inférieur à 18 ans.
\item En boxe anglaise la catégorie de poids pailles concerne les gens
dont le poids est inférieur à 47,128 kg (soit 105 livres, source :
\url{https://fr.wikipedia.org/wiki/Boxe\_anglaise} )
\item En France, seuls les gens qui ont moins de 10 777€ par an sont
exonérés d'impôt (0\%), ensuite de 10 778 à 27 478 c'est 11\%,
puis 30\% de 27 479 à 78 570, puis 41\% de 78 571 à 168 994 et
enfin 45\% pour ceux qui ont plus de 168 994 (source :
\url{https://www.service-public.fr/particuliers/vosdroits/F1419} )
\item La limite autorisée du taux d'alcool dans le sang par la loi en
2023 est de 0,5 g/L soit en équivalent 0,25 mg par litre d'air
expiré (source :
\url{https://www.legipermis.com/infractions/alcool-permis-conduire.html}
)
\item L'IMC (Indice de Masse Corporelle) permet d'établir des
catégories pour mesurer l'obésité avec par exemple le début de la
surcharge pondérale (surpoids) à partir de 25 kg/m\textsuperscript{2} (source :
\url{https://fr.wikipedia.org/wiki/Indice\_de\_masse\_corporelle} )
\end{enumerate}

Traduisons ces exemples concrets en inégalités mathématiques. Dans
chaque cas on utilisera la lettre \(x\) pour désigner la variable
inconnue.

\begin{enumerate}
\item \(x \geq 14\)
\item \(12 \leq x \leq 25\)
\item \(0 < x < 18\)
\item \(0 < x < 47,128\)
\item \(0 \leq x \leq 10 777\)
\item \(0 \leq x < 0,5\)
\item \(x \geq 25\)
\end{enumerate}

Ces inégalités peuvent être représentées graphiquement. On peut les
interpréter comme des zones délimitées par des bornes inférieures
et supérieures. Lorsqu'il n'y a qu'une seule borne apparente alors
ça veut dire que l'autre est du type \(\infty\) avec le signe
adéquat.

Taduisons ces inégalités en intervalles :

\begin{enumerate}
\item \(x \geq 14 \iff x\in [14; +\infty [\) on parle d'intervalle fermé.
\item \(12 \leq x \leq 25\iff x\in [12; 25]\) on parle d'intervalle fermé.
\item \(0 < x < 18\iff x\in ]0; 18[\) on parle d'intervalle ouvert.
\item \(0 < x < 47,128\iff x\in ]0; 47,128[\) on parle d'intervalle
ouvert.
\item \(0\leq x \leq 10 777\iff x\in [0; 10777]\) on parle d'intervalle
fermé.
\item \(0 \leq x < 0,5\iff x\in [0; 0,5[\) on parle d'intervalle ouvert à droite
\item \(x \geq 25\iff \in [25; +\infty[\) on parle d'intervalle fermé.
\end{enumerate}


Alors ici il s'agissait de modéliser à partir du monde réel donc il
y a des bornes implicites qui sont apparues :
\begin{enumerate}
\item Jusqu'à présent personne n'a dépassé le record de longévité de
Jeanne Calmant qui a vécu 122 ans. Mais comme on je ne peux pas
prédire l'avenir alors j'ai décidé de mettre la borne \(+\infty\)
même si c'est une borne théorique. Néanmoins, certains ont
prédit \href{https://amzn.to/3OKi2os}{la mort de la mort} (c'est le titre d'un livre de Laurent
Alexandre).
\item Ici les bornes étaient claires.
\item Là j'ai considéré que l'âge zéro n'est jamais atteint puisque
dès l'instant où l'on naît le temps s'écoule. Mais au regard de
l'exemple j'aurais même pu mettre une borne inférieure plus
grande.
\item Dans cette exemple c'est la même idée, personne n'a un poids de
zéro et encore moins quand on pratique la boxe anglaise.
\item Pour les revenus c'est malheureusement différent, il est hélas
possible d'avoir zéro revenu. Et pour information, 10 777 par an
ça fait moins de 900€ par mois (environ 898,08).
\item Les gens qui ne boivent pas d'alcool peuvent avoir un taux de
zéro gramme dans le sang.
\item Comme pour l'âge, le poids maximal n'est pas figé, d'ailleurs on
devrait parler de masse parce que le poids est une force qui
dépend du champ gravitationnel (la pesanteur) et donc il peut
changer en fonction de l'endroit dans l'espace.
\end{enumerate}

Ces exemples concrets avaient pour but de vous montrer pourquoi ce
principe de représentation d'ensembles de nombres réels ordonnés
peut avoir un intérêt et une utilité.

Les différents types d'intervalles de \(\mathbb{R}\) sont les
suivants :
\begin{enumerate}
\item Les intervalles ouverts à bornes finies \(]a ; b[\)
\item Les intervalles ouverts à gauche à bornes finies \(]a ; b]\) (on
peut aussi les appeler fermés à droite).
\item Les intervalles ouverts à droite à bornes finies \([a ; b[\) (on
peut aussi les appeler fermés à gauche).
\item Les intervalles ouverts à borne infinie à gauche \(]-\infty ; b[\)
\item Les intervalles ouverts à borne infinie à droite \(]a ; +\infty[\)
\item Les intervalles fermés à borne infinie à gauche \(]-\infty ; b]\)
\item Les intervalles fermés à borne infinie à droite \([a ; +\infty[\)
\item Les intervalles fermés à bornes finies \([a ; b]\)
\item L'intervalle ouvert \textbf{ET} fermé de l'ensemble des réels tout
entier \(\mathbb{R} = ]-\infty ; +\infty[\)
\end{enumerate}

Donnons quelques exemples :
\begin{align}
\frac{1}{2} &\in ]0 ; 1[ \\
1 &\in ]0 ; 1] \\
-\frac{1}{2} &\in [-0,5 ; 0[\\
-1 &\in ]-\infty ; 1[\\
\frac{2}{3} &\in ]0 ; +\infty[ \\
\pi &\in [3 ; +\infty[ \\
\Phi = \frac{1 + \sqrt{5}}{2} &\in [1 ; 2] \\
\tau = 2\pi &\in \mathbb{R}
\end{align}

On peut remarquer que l'ensemble des nombres entiers naturels est
inclus dans l'ensemble des nombres réels positifs :
\[\mathbb{N}\subset \mathbb{R}_{+} = [0; +\infty[\]

Par contre entre deux entiers naturels il y a une infinité de réels
et pour les représenter on utilise un intervalle.

Prenons par exemple les deux plus petits entiers naturels zéro et
un. On peut construire un intervalle \(I_0 = [0 ; 1]\) qui les contient
tous les deux. Le centre de cet intervalle \(I_0\) est un nombre décimal
\(\frac{1}{2} = 0,5\). Construisons un nouvel interval \(I_1 = [0;
   0,5]\). Le centre de cet intervalle \(I_1\) est un nombre décimal
\(\dfrac{1}{4} = 0,25\). On pourrait continuer ainsi à l'infini en
construisant des intervalles de la forme \[I_n = \left[0 ;
   \dfrac{1}{2^n}\right]\]

Mais si au lieu de prendre le centre on prenait le premier tiers ?
Et bien cette fois on aurait une borne supérieure qui serait un
nombre rationnel. \[J_n = \left[0 ; \dfrac{1}{3^n}\right]\]

Et si au lieu de prendre le tiers on divisait par le nombre \(\pi\) ?
Et bien cette fois on aurait une borne supérieure qui serait un
nombre réel. \[K_n = \left[0 ; \dfrac{1}{\pi^n}\right]\]

Bon dans ce dernier cas j'ai un peu triché parce que \(\pi\) est déjà
un réel alors que pours les \(I_n\) et les \(J_n\) on a pu faire la
construction uniquement en utilisant des entiers naturels.

Prenons un peu la mesure de ce que nous faisons. L'intervalle \([0 ;
   1]\) a pour longueur \(1\) de même que l'intervalle \([1 ; 2]\). Mais
qu'en est-il de l'intervalle \([-1 ; 0]\) ? \([-1 ; 1]\) ? Et de façon
générale, comment mesure-t-on la longueur d'un intervalle ?

C'est très simple, un intervalle étant un ensemble ordonné, on
mesure sa longueur en soustrayant la borne inférieure à la borne
supérieure (que l'intervalle soit ouvert ou fermé).

Il y a donc deux cas de figure :
\begin{enumerate}
\item Les bornes \(a\) et \(b\) sont finies et dans ce cas c'est juste une
soustraction normale \(|I| = |b - a|\).
\item Au moins l'une des bornes est infinie et dans ce cas \(|I| =
      +\infty\).
\end{enumerate}

Voici quelques exemples :
\begin{align}
\lvert ]-3 ; 3] \rvert &= 3 - (-3) = 3 + 3 = 6\\
\lvert [-5 ; 4] \rvert &= 5 - (-4) = 5 + 4 = 9\\
\lvert ]-\infty ; 23] \rvert &= +\infty\\
\lvert [7 ; +\infty[ \rvert &= +\infty\\
\lvert ]-\infty ; +\infty] \rvert &= +\infty
\end{align}

La longueur d'un intervalle est aussi appelée son amplitude. Par
exemple le nombre \(\pi\) appartient à l'intervalle \([3,14 ; 3,15]\)
qui a pour amplitude \(10^{-2} = 0,01\). Dit autrement, le nombre
\(\pi\) est compris entre \(3,14\) et \(3,15\) à \(10^{-2}\) près.

Faisons un petit exercice. Pour les nombres suivants trouver des
intervalles d'amplitude 1 puis 2 et enfin 3 auquels ils appartiennent :
\begin{enumerate}
\item \(\dfrac{1}{3}\)
\item \(\sqrt{2}\)
\item \(\Phi = \dfrac{1 + \sqrt{5}}{2}\)
\item \(\pi\)
\item \(\tau = 2\pi\)
\end{enumerate}

Jouez le jeu en cherchant par vous même vous progresserez alors que
si vous regardez la solution directement il vous progression sera
nulle.

\begin{enumerate}
\item À l'aide d'une machine calculez les décimales du nombre d'or
\[\Phi = \dfrac{1 + \sqrt{5}}{2}\]
\item Considérez la suite de nombres : 1, 1, 2, 3, 5, 8, 13, 21, 34,
55, 89,\ldots{}
\item Établir la règle de construction de la suite.
\item Calculer les quotients successifs en prenant comme numérateur le
suivant et comme dénominateur le prédécesseur ainsi
\(\dfrac{1}{1}, \dfrac{2}{1}, \dfrac{3}{2},\dots\).
\item La suite de nombre de la question 2 est connue sous le nom de
\href{https://fr.wikipedia.org/wiki/Suite\_de\_Fibonacci}{suite de Fibonacci} (mathématicien italien du XIIème siècle qui a
appris l'usage des chiffres indo-arabes grâce à son père qui
commerçait avec des marchands arabes près d'Alger). Combien de
termes de cette suite faut-il calculer pour approcher le nombre
d'or à \(10^{-5}\) près ? Vous donnerez les intervalles successifs.
\end{enumerate}

Faisons un petit jeu. Voici une série de nombres particuliers :
\begin{enumerate}
\item \(u = 0,1\dots\) avec une infinité de 1
\item \(v = 0,12\dots\) avec une infinité de 12
\item \(w = 0,123\dots\) avec une infinité de 123
\item \(x = 0,1234\dots\) avec une infinité de 1234
\item \(y = 0,12345\dots\) avec une infinité de 12345
\end{enumerate}

Pour chacun de ces nombres on va donner un intervalle d'amplitude
égale à la position de la dernière décimale distincte (donc 0,1
pour \(u\), 0,01 pour \(v\), 0,001 pour \(w\) et ainsi de suite) et
centré en ce nombre. Puis on va déterminer la fraction qui le
définit.

Voilà pour la première partie de la question

\begin{enumerate}
\item \(u = 0,1\dots\in I_u = [0,09 ; 0,19]\) en effet \(0,19 - 0,09 = 0,1\)
\item \(v = 0,12\dots\in I_v = [0,118 ; 0,128]\) en effet \(0,128 - 0,118 =
      0,01\)
\item \(w = 0,123\dots\in I_w = [0,1227 ; 0,1237]\) en effet \(0,1237 - 0,1227
      = 0,001\)
\item \(x = 0,1234\dots\in I_x = [0,12336 ; 0,12346]\) en effet \(0,12346 - 0,12336
      = 0,0001\)
\item \(y = 0,12345\dots\in I_y = [0,123455 ; 0,123465]\) en effet \(0,123465 - 0,123455
      = 0,0001\)
\end{enumerate}

Maintenant occupons-nous de la seconde partie.

\begin{enumerate}
\item \(10u = 1,1\dots\) donc \(10u - u = 1\) d'où \[u = \dfrac{1}{9}\in
      I_u = [0,09 ; 0,19]\]
\item \(100v = 12,12\dots\) donc \(100v - v = 12\) d'où \[v = \dfrac{12}{99}
      = \dfrac{4}{33}\in I_v = [0,118 ; 0,128]\]
\item \(1000w = 123,123\dots\) donc \(1000w - w = 123\) d'où \[w =
      \dfrac{123}{999} = \dfrac{41}{333}\in I_w = [0,1227 ; 0,1237]\]
\item \(10000x = 1234,1234\dots\) donc \(10000x - x = 1234\) d'où \[x =
      \dfrac{1234}{9999}\in I_x = [0,12336 ; 0,12346]\]
\item \(100000y = 12345,12345\dots\) donc \(100000y - y = 12345\) d'où \[y
      = \dfrac{12345}{99999}\in I_y = [0,123455 ; 0,123465]\]
\end{enumerate}


\subsection{Notation |a|. Distance entre deux nombres réels.}
\label{sec:org694ddab}
\subsection{Représentation de l'intervalle \([a - r, a + r]\) puis caractérisation par la condition \(|x - a| \leq r\)}
\label{sec:org1e0f28b}
\subsection{Ensemble \(\mathbb{D}\) des nombres décimaux. Encadrement décimal d'un nombre réel à \(10^{-n}\) près}
\label{sec:org6416331}
\subsection{Ensemble \(\mathbb{Q}\) des nombres rationnels. Nombres irrationnels ; exemples fournis par la géométrie, par exemple \(\sqrt{2}\) et \(\pi\)}
\label{sec:org8081ab6}

\section{Capacités attendues}
\label{sec:orgb7bb030}
\section{Démonstrations}
\label{sec:org11deca3}
\section{Exemple d'algorithme}
\label{sec:org8803207}
\section{Approfondissements possibles}
\label{sec:org8396eef}
\section{Après ce livre}
\label{sec:org93bb440}
\end{document}